\documentclass[12pt]{article}

\usepackage[a4paper, margin=1in]{geometry}
\usepackage{hyperref}
\usepackage{enumitem}
\usepackage{longtable}

\title{Inventory System Project Report}
\author{Inventory Management System -- Data Producer Module}
\date{\today}

\begin{document}

\maketitle
\tableofcontents
\newpage

\section{Introduction}

This document describes the structure and functionality of a Java-based Inventory System developed as part of a modular enterprise architecture. The current implementation represents the \textbf{Data Producer layer}, responsible for reading product data from CSV files, storing it in memory, performing calculations, and preparing it for persistence in a relational database through JDBC.

The system demonstrates object-oriented principles including abstraction, inheritance, polymorphism, and separation of concerns.

\section{Project Architecture Overview}

The system is organized into the following logical components:

\begin{itemize}
    \item \textbf{Model Layer} -- Defines the product hierarchy.
    \item \textbf{I/O Layer} -- Handles CSV data ingestion.
    \item \textbf{Application Layer (Main)} -- Provides console interaction and business logic.
    \item \textbf{Data Consumer Layer (JDBC)} -- Responsible for database persistence (external to this report).
\end{itemize}

\section{Model Layer}

\subsection{Abstract Class: Product}

The \texttt{Product} class represents a generic inventory item and defines attributes common to all product types:

\begin{itemize}
    \item Category
    \item Product Name
    \item Quantity
    \item Unit Cost
    \item Margin
\end{itemize}

It provides shared behavior:

\begin{itemize}
    \item Inventory value calculation
    \item Abstract unit price calculation
    \item Abstract packing/shipping cost calculation
\end{itemize}

This design enforces a contract requiring subclasses to implement pricing and packing logic.

\subsection{Subclass: Electronics}

The \texttt{Electronics} class extends \texttt{Product} and represents electronic goods.

Responsibilities:

\begin{itemize}
    \item Implements pricing based on margin.
    \item Implements packing cost based on weight.
\end{itemize}

The overridden \texttt{getShippingCost(double weight)} method demonstrates subclass-specific behavior by computing packing requirements using a weight parameter.

\subsection{Subclass: Clothing}

The \texttt{Clothing} class extends \texttt{Product} and represents apparel products.

Responsibilities:

\begin{itemize}
    \item Implements pricing based on margin.
    \item Implements packing cost based on volume.
\end{itemize}

The overridden \texttt{getShippingCost(double volume)} method demonstrates polymorphism through volume-based packing calculation.

\section{Demonstration of Polymorphism}

The system uses polymorphism through:

\begin{itemize}
    \item A shared \texttt{List<Product>} collection.
    \item Overridden implementations of \texttt{getUnitPrice()}.
    \item Overridden implementations of \texttt{getShippingCost(double)}.
\end{itemize}

At runtime, the correct subclass method is invoked depending on the actual object type, demonstrating dynamic method dispatch.

\section{I/O Layer: CSVProductReader}

The \texttt{CSVProductReader} class is responsible for:

\begin{itemize}
    \item Reading CSV files from the resources directory.
    \item Validating mandatory fields.
    \item Parsing numeric values safely.
    \item Instantiating appropriate subclass objects.
    \item Returning lists of \texttt{Product}.
\end{itemize}

This class isolates file parsing logic from business logic, improving maintainability and modularity.

\section{Application Layer (Main)}

The \texttt{Main} class provides a console-based interface with options to:

\begin{enumerate}
    \item Import Electronics CSV
    \item Import Clothing CSV
    \item Display Electronics
    \item Display Clothing
    \item Compute Total Inventory Value
    \item Display Packing Costs
    \item Exit
\end{enumerate}

Packing costs are calculated by requesting per-unit weight or volume from the system administrator. This demonstrates how subclass-specific parameters influence overridden methods without modifying shared data structures.

\section{Design Decisions}

\subsection{Minimal Intrusion Principle}

The project intentionally avoids modifying:

\begin{itemize}
    \item CSV parsing logic
    \item Existing constructors
    \item Shared in-memory product structures
\end{itemize}

This ensures compatibility with downstream JDBC persistence logic.

\subsection{Separation of Concerns}

\begin{itemize}
    \item Model classes handle domain logic.
    \item I/O classes handle file ingestion.
    \item Main handles interaction and orchestration.
    \item JDBC layer handles persistence.
\end{itemize}

\subsection{Extensibility}

The design supports future enhancements:

\begin{itemize}
    \item Storing actual weight/volume per product.
    \item Implementing real shipping calculations.
    \item Adding additional product subclasses.
    \item Integrating with a warehouse management system.
\end{itemize}

\section{Role Within a Larger Inventory System}

Within an enterprise architecture, this module functions as:

\begin{itemize}
    \item A \textbf{data ingestion component}.
    \item A \textbf{business rule processor}.
    \item A \textbf{pre-persistence validation layer}.
\end{itemize}

It prepares structured, validated product objects for insertion into an SQL database through JDBC, forming part of a multi-tier inventory management system.

\section{Future Work: Toward a Full Inventory Simulation System}

The current implementation represents a foundational data ingestion and processing module within a broader inventory architecture. While it successfully demonstrates object-oriented design, CSV ingestion, and preparation for database persistence, several extensions would transform it into a fully operational inventory management and simulation system.

\subsection{Inventory State Management}

Future versions should support dynamic stock movement rather than static inventory records. This includes:

\begin{itemize}
    \item Stock In (receiving goods from suppliers)
    \item Stock Out (sales or shipments to customers)
    \item Inventory adjustments (corrections, damages, audits)
\end{itemize}

This would require controlled methods for increasing and decreasing product quantities with validation to prevent negative inventory levels.

\subsection{Transaction Modeling}

A realistic inventory system must record the reason behind quantity changes. Introducing an \texttt{InventoryTransaction} class would enable:

\begin{itemize}
    \item Transaction type tracking (IN / OUT / ADJUSTMENT)
    \item Timestamp recording
    \item Supplier or customer reference
    \item Historical auditing
\end{itemize}

This enables traceability and supports reporting and compliance requirements.

\subsection{Service Layer Introduction}

Currently, the main application coordinates logic directly. A future architectural improvement would introduce an \texttt{InventoryService} layer responsible for:

\begin{itemize}
    \item Processing transactions
    \item Validating stock levels
    \item Applying business rules
    \item Triggering low-stock alerts
\end{itemize}

This improves separation of concerns and scalability.

\subsection{Warehouse and Location Modeling}

To simulate real-world logistics, the system could introduce:

\begin{itemize}
    \item Multiple warehouses
    \item Storage locations or bins
    \item Capacity constraints
    \item Volume-based storage optimization
\end{itemize}

This allows inventory to be tracked not only by product but also by physical location.

\subsection{Reorder and Safety Stock Logic}

A practical inventory system typically includes:

\begin{itemize}
    \item Reorder points
    \item Safety stock levels
    \item Automatic purchase order generation
    \item Lead time simulation
\end{itemize}

These features help prevent stockouts and optimize supply chain efficiency.

\subsection{Inventory Metrics and Reporting}

Future enhancements may compute operational metrics such as:

\begin{itemize}
    \item Inventory turnover rate
    \item Days of inventory on hand
    \item Fast-moving and slow-moving items
    \item Dead stock analysis
    \item Historical inventory valuation
\end{itemize}

These analytical tools provide strategic business insight.

\subsection{Simulation Engine}

To evolve into a full inventory simulator, an event-driven engine could be implemented to model:

\begin{itemize}
    \item Random customer demand
    \item Supplier lead times
    \item Seasonal demand variations
    \item Backorders and stockouts
\end{itemize}

A simulation loop operating over simulated time periods (e.g., days or weeks) would allow the study of system behavior under varying operational conditions.

\subsection{Advanced Extensions}

Additional long-term enhancements may include:

\begin{itemize}
    \item FIFO/LIFO inventory valuation
    \item Batch tracking
    \item Expiration date management
    \item Returns processing
    \item Role-based access control
    \item Graphical user interface integration
\end{itemize}

\subsection{Architectural Evolution}

With these enhancements, the system would evolve from a static data ingestion module into a layered architecture consisting of:

\begin{itemize}
    \item Presentation Layer (Console or GUI)
    \item Service Layer (Business Logic)
    \item Domain Model (Products, Transactions, Warehouses)
    \item Persistence Layer (JDBC / SQL Database)
    \item Simulation Engine (Optional Analytical Component)
\end{itemize}

Such an evolution would position the project as a fully functional inventory management and simulation platform suitable for academic research, enterprise prototyping, or operational training environments.



\section{Conclusion}

The Inventory System demonstrates key object-oriented principles while maintaining architectural stability. The addition of subclass-specific packing calculations illustrates polymorphism without disrupting the data production workflow.

The system is modular, extensible, and suitable for integration into a broader enterprise inventory solution.

\end{document}
